\documentclass[]{article}
\usepackage{lmodern}
\usepackage{amssymb,amsmath}
\usepackage{ifxetex,ifluatex}
\usepackage{fixltx2e} % provides \textsubscript
\ifnum 0\ifxetex 1\fi\ifluatex 1\fi=0 % if pdftex
  \usepackage[T1]{fontenc}
  \usepackage[utf8]{inputenc}
\else % if luatex or xelatex
  \ifxetex
    \usepackage{mathspec}
  \else
    \usepackage{fontspec}
  \fi
  \defaultfontfeatures{Ligatures=TeX,Scale=MatchLowercase}
\fi
% use upquote if available, for straight quotes in verbatim environments
\IfFileExists{upquote.sty}{\usepackage{upquote}}{}
% use microtype if available
\IfFileExists{microtype.sty}{%
\usepackage{microtype}
\UseMicrotypeSet[protrusion]{basicmath} % disable protrusion for tt fonts
}{}
\usepackage[margin=1.5cm]{geometry}
\usepackage{hyperref}
\hypersetup{unicode=true,
            pdftitle={Don't Break a Leg! Road Safety in the City of Toronto},
            pdfauthor={Sunwoo (Angela) Kang, Sergio E. Betancourt, Jing Li, and Jiahui (Eddy) Du},
            pdfborder={0 0 0},
            breaklinks=true}
\urlstyle{same}  % don't use monospace font for urls
\usepackage{color}
\usepackage{fancyvrb}
\newcommand{\VerbBar}{|}
\newcommand{\VERB}{\Verb[commandchars=\\\{\}]}
\DefineVerbatimEnvironment{Highlighting}{Verbatim}{commandchars=\\\{\}}
% Add ',fontsize=\small' for more characters per line
\usepackage{framed}
\definecolor{shadecolor}{RGB}{248,248,248}
\newenvironment{Shaded}{\begin{snugshade}}{\end{snugshade}}
\newcommand{\KeywordTok}[1]{\textcolor[rgb]{0.13,0.29,0.53}{\textbf{#1}}}
\newcommand{\DataTypeTok}[1]{\textcolor[rgb]{0.13,0.29,0.53}{#1}}
\newcommand{\DecValTok}[1]{\textcolor[rgb]{0.00,0.00,0.81}{#1}}
\newcommand{\BaseNTok}[1]{\textcolor[rgb]{0.00,0.00,0.81}{#1}}
\newcommand{\FloatTok}[1]{\textcolor[rgb]{0.00,0.00,0.81}{#1}}
\newcommand{\ConstantTok}[1]{\textcolor[rgb]{0.00,0.00,0.00}{#1}}
\newcommand{\CharTok}[1]{\textcolor[rgb]{0.31,0.60,0.02}{#1}}
\newcommand{\SpecialCharTok}[1]{\textcolor[rgb]{0.00,0.00,0.00}{#1}}
\newcommand{\StringTok}[1]{\textcolor[rgb]{0.31,0.60,0.02}{#1}}
\newcommand{\VerbatimStringTok}[1]{\textcolor[rgb]{0.31,0.60,0.02}{#1}}
\newcommand{\SpecialStringTok}[1]{\textcolor[rgb]{0.31,0.60,0.02}{#1}}
\newcommand{\ImportTok}[1]{#1}
\newcommand{\CommentTok}[1]{\textcolor[rgb]{0.56,0.35,0.01}{\textit{#1}}}
\newcommand{\DocumentationTok}[1]{\textcolor[rgb]{0.56,0.35,0.01}{\textbf{\textit{#1}}}}
\newcommand{\AnnotationTok}[1]{\textcolor[rgb]{0.56,0.35,0.01}{\textbf{\textit{#1}}}}
\newcommand{\CommentVarTok}[1]{\textcolor[rgb]{0.56,0.35,0.01}{\textbf{\textit{#1}}}}
\newcommand{\OtherTok}[1]{\textcolor[rgb]{0.56,0.35,0.01}{#1}}
\newcommand{\FunctionTok}[1]{\textcolor[rgb]{0.00,0.00,0.00}{#1}}
\newcommand{\VariableTok}[1]{\textcolor[rgb]{0.00,0.00,0.00}{#1}}
\newcommand{\ControlFlowTok}[1]{\textcolor[rgb]{0.13,0.29,0.53}{\textbf{#1}}}
\newcommand{\OperatorTok}[1]{\textcolor[rgb]{0.81,0.36,0.00}{\textbf{#1}}}
\newcommand{\BuiltInTok}[1]{#1}
\newcommand{\ExtensionTok}[1]{#1}
\newcommand{\PreprocessorTok}[1]{\textcolor[rgb]{0.56,0.35,0.01}{\textit{#1}}}
\newcommand{\AttributeTok}[1]{\textcolor[rgb]{0.77,0.63,0.00}{#1}}
\newcommand{\RegionMarkerTok}[1]{#1}
\newcommand{\InformationTok}[1]{\textcolor[rgb]{0.56,0.35,0.01}{\textbf{\textit{#1}}}}
\newcommand{\WarningTok}[1]{\textcolor[rgb]{0.56,0.35,0.01}{\textbf{\textit{#1}}}}
\newcommand{\AlertTok}[1]{\textcolor[rgb]{0.94,0.16,0.16}{#1}}
\newcommand{\ErrorTok}[1]{\textcolor[rgb]{0.64,0.00,0.00}{\textbf{#1}}}
\newcommand{\NormalTok}[1]{#1}
\usepackage{longtable,booktabs}
\usepackage{graphicx,grffile}
\makeatletter
\def\maxwidth{\ifdim\Gin@nat@width>\linewidth\linewidth\else\Gin@nat@width\fi}
\def\maxheight{\ifdim\Gin@nat@height>\textheight\textheight\else\Gin@nat@height\fi}
\makeatother
% Scale images if necessary, so that they will not overflow the page
% margins by default, and it is still possible to overwrite the defaults
% using explicit options in \includegraphics[width, height, ...]{}
\setkeys{Gin}{width=\maxwidth,height=\maxheight,keepaspectratio}
\IfFileExists{parskip.sty}{%
\usepackage{parskip}
}{% else
\setlength{\parindent}{0pt}
\setlength{\parskip}{6pt plus 2pt minus 1pt}
}
\setlength{\emergencystretch}{3em}  % prevent overfull lines
\providecommand{\tightlist}{%
  \setlength{\itemsep}{0pt}\setlength{\parskip}{0pt}}
\setcounter{secnumdepth}{0}
% Redefines (sub)paragraphs to behave more like sections
\ifx\paragraph\undefined\else
\let\oldparagraph\paragraph
\renewcommand{\paragraph}[1]{\oldparagraph{#1}\mbox{}}
\fi
\ifx\subparagraph\undefined\else
\let\oldsubparagraph\subparagraph
\renewcommand{\subparagraph}[1]{\oldsubparagraph{#1}\mbox{}}
\fi

%%% Use protect on footnotes to avoid problems with footnotes in titles
\let\rmarkdownfootnote\footnote%
\def\footnote{\protect\rmarkdownfootnote}

%%% Change title format to be more compact
\usepackage{titling}

% Create subtitle command for use in maketitle
\newcommand{\subtitle}[1]{
  \posttitle{
    \begin{center}\large#1\end{center}
    }
}

\setlength{\droptitle}{-2em}

  \title{Don't Break a Leg! Road Safety in the City of Toronto}
    \pretitle{\vspace{\droptitle}\centering\huge}
  \posttitle{\par}
  \subtitle{STA2453 - Project II Draft}
  \author{Sunwoo (Angela) Kang, Sergio E. Betancourt, Jing Li, and Jiahui (Eddy)
Du}
    \preauthor{\centering\large\emph}
  \postauthor{\par}
      \predate{\centering\large\emph}
  \postdate{\par}
    \date{2019-03-08}

\usepackage{booktabs}
\usepackage{longtable}
\usepackage{array}
\usepackage{multirow}
\usepackage[table]{xcolor}
\usepackage{wrapfig}
\usepackage{float}
\usepackage{colortbl}
\usepackage{pdflscape}
\usepackage{tabu}
\usepackage{threeparttable}
\usepackage{threeparttablex}
\usepackage[normalem]{ulem}
\usepackage{makecell}

\usepackage{titling}
\usepackage{setspace}\singlespacing
\usepackage{subfig}

\begin{document}
\maketitle

\section{Introduction}\label{introduction}

Road traffic safety is a crucial component of urban planning and
development. Nowadays governments (and sometimes the private sector)
dedicate significant resources to providing ample and sufficient
infrastructure to accommodate diverse modes of transportation, thereby
increasing the productivity of any given urban area. In this project we
examine road safety in the City of Toronto from 2007 to 2017 and explore
the areas with highest risk of a traffic incident, controlling for
different factors.

\section{Methods}\label{methods}

We define the City of Toronto as per the these guidelines
(\url{https://www.toronto.ca/city-government/data-research-maps/neighbourhoods-communities/neighbourhood-profiles/}).
Below are the neighborhood limits and the 2016 population estimates:

\begin{figure}[H]

{\centering \includegraphics[width=0.8\linewidth]{Toronto-2016} 

}

\caption{\label{fig:figs}EDA with regards to the City of Toronto}\label{fig:unnamed-chunk-2}
\end{figure}

\subsubsection{Primary Questions}\label{primary-questions}

The analysis focuses on answering two main questions:

\begin{enumerate}
  \item Which areas in Toronto are the most unsafe controlling for other factors?
  \item Which factors are related to the safety of areas?
\end{enumerate}

\subsubsection{Data Collection}\label{data-collection}

For our analysis we employed data from
\href{https://www.data.geotab.com}{Geotab}, the
\href{https://www.toronto.ca/city-government/data-research-maps}{City of
Toronto},
\href{http://climate.weather.gc.ca/historical_data/search_historic_data_e.html}{Environment
Canada}, and the \href{http://data.torontopolice.on.ca}{Toronto Police
Service}. Each of these datasets contains different levels of
granularity and information, and were therefore combined to obtain the
following variables of interest:

\begin{longtable}[]{@{}lll@{}}
\toprule
City of Toronto & Toronto Police & Geotab\tabularnewline
\midrule
\endhead
1st & kernel & num\_filters\tabularnewline
2nd & kernel & \(2 \times\)num\_filters\tabularnewline
3rd & kernel & \(2 \times\)num\_filters\tabularnewline
4th & kernel & num\_filters\tabularnewline
5th & kernel & 3\tabularnewline
6th & kernel & 3\tabularnewline
\bottomrule
\end{longtable}

\subsubsection{Data Preparation}\label{data-preparation}

The following diagram provides an overview of the merged data.

We first merge the Road Impediment and Hazardous Driving Areas (Geotab)
with the Toronto Police Service dataset by geohash. After merging, we
calculate the aggregate incidents counts by ward and month. And we take
the mean of acceleration, vehicle volumes and severity scores
information per ward. We then append the monthly weather information of
toronto area by month.

Ultimately we

\subsubsection{Modeling}\label{modeling}

We model our outcome of interest using a temporal model (to be expanded
to spatial in the following iteration).

\section{Results}\label{results}

\section{Conclusions and Discussion}\label{conclusions-and-discussion}

One of the biggest limitations in our project has been data quality and
granularity.

\section{Appendix: Dataset Variables and
Definitions}\label{appendix-dataset-variables-and-definitions}

\begin{tabular}{lll}
\toprule
Feature & Description & Source\\
\midrule
YEAR & Year in range (2007-2017) inclusive & Automobile (Toronto Police)\\
MONTH & Month in range 1-12 inclusive & Automobile (Toronto Police)\\
Ward\_ID & Ward in range (1-44) inclusive & Automobile (Toronto Police)\\
IncidentsTotal\_TP & Total number of incidents & Automobile (Toronto Police)\\
Dark & Count accidents ocurred on dark conditions & Automobile (Toronto Police)\\
Dawn & Count accidents ocurred on dawn conditions & Automobile (Toronto Police)\\
Daylight & Count accidents ocurred on daylight conditions & Automobile (Toronto Police)\\
Dusk & Count accidents ocurred on dusk conditions & Automobile (Toronto Police)\\
Inv\_PED & Count accidents involved pedstrains & Automobile (Toronto Police)\\
Inv\_CYC & Count accidents involved cyclists & Automobile (Toronto Police)\\
Inv\_AM & Count accidents involved automobiles & Automobile (Toronto Police)\\
Inv\_MC & Count accidents involved motorcycles & Automobile (Toronto Police)\\
Inv\_TC & Count accidents involved trucks & Automobile (Toronto Police)\\
Speeding & Count accidents ocurred on speeding condition & Automobile (Toronto Police)\\
Ag\_Driv & Count accidents ocurred on angry driving condition & Automobile (Toronto Police)\\
Redlight & Count accidents ocurred with redlight & Automobile (Toronto Police)\\
Alcohol & Count accidents ocurred with driver with alcohol & Automobile (Toronto Police)\\
Disability & Count accidents ocurred with driver with disability & Automobile (Toronto Police)\\
SeverityScore & Average Score of Severitylevel (harsh brake) & HDA(Geotab)\\
IncidentsTotal\_Geotab & Monthly average of total number of incidents & HDA(Geotab)\\
AvgAcceleration & Monthly average acceleration & RI(Geotab)\\
PercentOfVehicles & Monthly average on percentage of vehicles & RI(Geotab)\\
AvgMonthlyVolume & Monthly average on vehicle volumes & RI(Geotab)\\
PercentCar & Monthly average on car percentage & RI(Geotab)\\
PercentMPV & Monthly average on MPV percentage & RI(Geotab)\\
PercentLDT & Monthly average on LDT percentage & RI(Geotab)\\
PercentMDT & Monthly average on MDT percentage & RI(Geotab)\\
PercentHDT & Monthly average on HDT percentage & RI(Geotab)\\
PercentOther & Monthly average on other vehicle percentage & RI(Geotab)\\
Daily\_dif & Monthly average on daily Weather change (in celsus) & Weather\\
Max\_Temp & Monthly max on highest daily Weather degree (celsus) & Weather\\
Min\_Temp & Monthly min on lowest daily Weather degree (celsus) & Weather\\
Ave\_Temp & Monthly average on daily average Weather (in celsus) & Weather\\
Rain\_vol & Monthly average on daily rain volumn & Weather\\
Snow\_vol & Monthly average on daily snow volumn & Weather\\
\bottomrule
\end{tabular}

\pagebreak

\newpage

\section{Appendix: Code}\label{appendix-code}

\begin{Shaded}
\begin{Highlighting}[]
\KeywordTok{library}\NormalTok{(MASS); }\KeywordTok{library}\NormalTok{(lmtest); }\KeywordTok{library}\NormalTok{(knitr); }\KeywordTok{library}\NormalTok{(kableExtra); }\KeywordTok{library}\NormalTok{(nleqslv);}
\KeywordTok{library}\NormalTok{(Pmisc); }\KeywordTok{library}\NormalTok{(extrafont); }\KeywordTok{library}\NormalTok{(VGAM); }\KeywordTok{library}\NormalTok{(INLA); }\KeywordTok{library}\NormalTok{(MEMSS);}
\KeywordTok{library}\NormalTok{(nlme); }\KeywordTok{library}\NormalTok{(ciTools); }\KeywordTok{library}\NormalTok{(tibble); }\KeywordTok{library}\NormalTok{(sp); }\KeywordTok{library}\NormalTok{(dplyr);}
\NormalTok{knitr}\OperatorTok{::}\NormalTok{opts_chunk}\OperatorTok{$}\KeywordTok{set}\NormalTok{(}\DataTypeTok{fig.pos =} \StringTok{'H'}\NormalTok{);}
\CommentTok{# Loading polygon and population data from the City of Toronto}
\NormalTok{population <-}\StringTok{ }\KeywordTok{read.csv}\NormalTok{(}\StringTok{"https://raw.githubusercontent.com/sergiosonline/data_sci_geo/master/data/neighbourhoods_planning_areas_wgs84_SEB/Wellbeing_TO_2016%20Census_Total%20Pop_Total%20Change_Age%20Groups.csv"}\NormalTok{,}\DataTypeTok{stringsAsFactors =} \OtherTok{FALSE}\NormalTok{,}\DataTypeTok{header=}\NormalTok{T)}

\KeywordTok{require}\NormalTok{(sf)}
\NormalTok{shape <-}\StringTok{ }\KeywordTok{read_sf}\NormalTok{(}\DataTypeTok{dsn =} \StringTok{"https://github.com/sergiosonline/data_sci_geo/blob/master/data/neighbourhoods_planning_areas_wgs84_SEB/NEIGHBORHOODS_WGS84.shp?raw=true"}\NormalTok{, }\DataTypeTok{layer =} \StringTok{"NEIGHBORHOODS_WGS84"}\NormalTok{)}

\NormalTok{neighborhoods <-}\StringTok{ }\NormalTok{shape}

\CommentTok{# Adding populaation info to neighborhood polygon}
\NormalTok{neighborhoods <-}\StringTok{ }\KeywordTok{add_column}\NormalTok{(neighborhoods, }\StringTok{'2016pop'}\NormalTok{=}\OtherTok{NA}\NormalTok{, }\StringTok{'x_coords'}\NormalTok{ =}\StringTok{ }\OtherTok{NA}\NormalTok{, }\StringTok{'y_coords'}\NormalTok{ =}\StringTok{ }\OtherTok{NA}\NormalTok{)}

\CommentTok{# Separating X and Y coordinates from polygon}
\ControlFlowTok{for}\NormalTok{ (hood }\ControlFlowTok{in}\NormalTok{ neighborhoods}\OperatorTok{$}\NormalTok{AREA_NAME) \{}
\NormalTok{  ## Adding population}
\NormalTok{  pop =}\StringTok{ }\KeywordTok{as.numeric}\NormalTok{(neighborhoods[neighborhoods}\OperatorTok{$}\NormalTok{AREA_NAME }\OperatorTok{==}\StringTok{ }\NormalTok{hood,][[}\StringTok{"AREA_S_CD"}\NormalTok{]])}
\NormalTok{  neighborhoods[neighborhoods}\OperatorTok{$}\NormalTok{AREA_NAME }\OperatorTok{==}\StringTok{ }\NormalTok{hood,]}\OperatorTok{$}\StringTok{'2016pop'}\NormalTok{ =}\StringTok{ }
\StringTok{    }\NormalTok{population[population}\OperatorTok{$}\NormalTok{HoodID }\OperatorTok{==}\StringTok{ }\NormalTok{pop,]}\OperatorTok{$}\NormalTok{Pop2016}
\NormalTok{  ## Adding x-y}
\NormalTok{  temp =}\StringTok{ }\KeywordTok{unlist}\NormalTok{(}\KeywordTok{subset}\NormalTok{(neighborhoods,AREA_NAME }\OperatorTok{==}\StringTok{ }\NormalTok{hood)}\OperatorTok{$}\NormalTok{geometry[[}\DecValTok{1}\NormalTok{]])}
\NormalTok{  ll =}\StringTok{ }\KeywordTok{length}\NormalTok{(temp)}
\NormalTok{  x_coord =}\StringTok{ }\KeywordTok{list}\NormalTok{(temp[}\DecValTok{1}\OperatorTok{:}\NormalTok{(ll}\OperatorTok{/}\DecValTok{2}\NormalTok{)])}
\NormalTok{  y_coord =}\StringTok{ }\KeywordTok{list}\NormalTok{(temp[((ll}\OperatorTok{/}\DecValTok{2}\NormalTok{)}\OperatorTok{+}\DecValTok{1}\NormalTok{)}\OperatorTok{:}\NormalTok{ll])}
\NormalTok{  neighborhoods[neighborhoods}\OperatorTok{$}\NormalTok{AREA_NAME }\OperatorTok{==}\StringTok{ }\NormalTok{hood,]}\OperatorTok{$}\NormalTok{x_coords =}\StringTok{ }\NormalTok{x_coord}
\NormalTok{  neighborhoods[neighborhoods}\OperatorTok{$}\NormalTok{AREA_NAME }\OperatorTok{==}\StringTok{ }\NormalTok{hood,]}\OperatorTok{$}\NormalTok{y_coords =}\StringTok{ }\NormalTok{y_coord}
\NormalTok{\}}

\KeywordTok{st_write}\NormalTok{(neighborhoods,}\StringTok{"~/Desktop/Grad_School/COURSEWORK/Spring 2019/Data Science/rough work/neighbourhoods_planning_areas_wgs84_SEB/NEIGHBORHOODS_WGS84.shp"}\NormalTok{,}
\NormalTok{         , }\DataTypeTok{delete_layer =} \OtherTok{TRUE}\NormalTok{)}

\KeywordTok{require}\NormalTok{(sf)}
\NormalTok{neighborhoods <-}\StringTok{ }\KeywordTok{read_sf}\NormalTok{(}\DataTypeTok{dsn =} \StringTok{"~/Desktop/Grad_School/COURSEWORK/Spring 2019/Data Science/rough work/neighbourhoods_planning_areas_wgs84_SEB/"}\NormalTok{, }\DataTypeTok{layer =} \StringTok{"NEIGHBORHOODS_WGS84"}\NormalTok{)}

\KeywordTok{plot}\NormalTok{(neighborhoods)}

\NormalTok{## Visualizing above polygon, after customization}
\NormalTok{url0 <-}\StringTok{ "https://raw.githubusercontent.com/sergiosonline/data_sci_geo/master/reports/draft/STA2453-Toronto-2016.png"}
\KeywordTok{download.file}\NormalTok{(}\DataTypeTok{url =}\NormalTok{ url0,}
          \DataTypeTok{destfile =} \StringTok{"toronto-population.png"}\NormalTok{,}
          \DataTypeTok{mode =} \StringTok{'wb'}\NormalTok{)}

\CommentTok{# Visualization of fatal vehicular incidents in the City of Toronto 2010-2016}
\NormalTok{collisiondat <-}\StringTok{ }\KeywordTok{read.csv}\NormalTok{(}\StringTok{"https://raw.githubusercontent.com/sergiosonline/data_sci_geo/master/data/Fatal_Collisions.csv"}\NormalTok{, }\DataTypeTok{header=}\NormalTok{T, }\DataTypeTok{stringsAsFactors =} \OtherTok{FALSE}\NormalTok{)}

\KeywordTok{coordinates}\NormalTok{(collisiondat) <-}\StringTok{ }\ErrorTok{~}\NormalTok{LONGITUDE}\OperatorTok{+}\NormalTok{LATITUDE}
\CommentTok{#4326 - WGS84 std}
\KeywordTok{proj4string}\NormalTok{(collisiondat) <-}\StringTok{ "+init=epsg:3034"} \CommentTok{#"+init=epsg:4326" }
\NormalTok{data_L93 <-}\StringTok{ }\KeywordTok{spTransform}\NormalTok{(collisiondat, }\KeywordTok{CRS}\NormalTok{(}\StringTok{"+proj=lcc +lat_1=44 +lat_2=49 +lat_0=46.5 +lon_0=3 +x_0=490000 +y_0=4620000 +ellps=GRS80 +units=m +no_defs"}\NormalTok{))}
\CommentTok{#x_0/y_0 = 0.1060606}

\NormalTok{data_L93 <-}\StringTok{ }\KeywordTok{spTransform}\NormalTok{(collisiondat, }\KeywordTok{CRS}\NormalTok{(}\StringTok{"+proj=longlat +lat_1=44 +lat_2=49 +lat_0=46.5 +lon_0=3 +x_0=490000 +y_0=4620000 +ellps=GRS80 +units=m +no_defs"}\NormalTok{))}

\NormalTok{url1 <-}\StringTok{ "https://raw.githubusercontent.com/sergiosonline/data_sci_geo/master/reports/draft/STA2453-Toronto-2016.png"}
\KeywordTok{download.file}\NormalTok{(}\DataTypeTok{url =}\NormalTok{ url0,}
          \DataTypeTok{destfile =} \StringTok{"toronto_incidents.png"}\NormalTok{,}
          \DataTypeTok{mode =} \StringTok{'wb'}\NormalTok{)}

\NormalTok{knitr}\OperatorTok{::}\KeywordTok{include_graphics}\NormalTok{(}\DataTypeTok{path=}\StringTok{"Toronto-2016.png"}\NormalTok{)}

\CommentTok{#spTransform() #Transform polygon or raster into Euclidian object - 3026 is Google std}
\CommentTok{# Loading final monthly incident data, by neighborhood}
\NormalTok{incidentdat <-}\StringTok{ }\KeywordTok{read.csv}\NormalTok{(}\StringTok{"https://raw.githubusercontent.com/sergiosonline/data_sci_geo/master/data/incident_by_year_month_ward_2Mar.csv"}\NormalTok{, }\DataTypeTok{header=}\NormalTok{T, }\DataTypeTok{stringsAsFactors =}\NormalTok{ F)}

\NormalTok{var_def <-}\StringTok{ }\KeywordTok{read.csv}\NormalTok{(}\StringTok{"https://raw.githubusercontent.com/sergiosonline/data_sci_geo/master/reports/draft/variable_def.csv"}\NormalTok{,}\DataTypeTok{header=}\NormalTok{T, }\DataTypeTok{stringsAsFactors =}\NormalTok{ F, }\DataTypeTok{sep=}\StringTok{","}\NormalTok{)}

\NormalTok{knitr}\OperatorTok{::}\KeywordTok{kable}\NormalTok{(var_def, }\DataTypeTok{format=}\StringTok{"latex"}\NormalTok{, }\DataTypeTok{booktab=}\NormalTok{T, }\DataTypeTok{linesep =} \StringTok{""}\NormalTok{) }\CommentTok{#escape=F, }
\end{Highlighting}
\end{Shaded}


\end{document}
